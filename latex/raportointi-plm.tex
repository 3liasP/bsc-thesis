\chapter{Raportointi ja PLM järjestelmät} \label{Raportointi ja PLM järjestelmät}

\section{PLM-strategia ja PLM-järjestelmät lyhyesti} \label{PLM-strategia ja PLM-järjestelmät lyhyesti}

Laajempana käsitteenä tuotteen elinkaaren hallinta eli PLM voidaan nähdä yrityksen strategiana hallita tuotetietoja. PLM strategiana koostuu tuotteista, organisaatioista, työmenetelmistä, prosesseista, ihmisistä ja lopulta usein myös tietoteknisestä elinkaaren hallintajärjestelmästä.

Tuotteen elinkaari voidaan jakaa alku-, keski- ja loppuvaiheeseen. Tuotteen elinkaaren pääpiirteet on hyvä ymmärtää, jotta PLM-käsitettä voidaan tarkastella syvällisemmin. Bouhaddoun (2012) konferenssiartikkeli \textit{"PLM Model for Supply Chain Optimization"} määrittelee tuotteen elinkaaren vaiheet ja niiden piirteet seuraavasti:  \cite{bouhaddou_plm_2012}
\begin{itemize}
\item Alkuvaiheessa tuotteen vaatimuksia määritellään ja tuote on luonnosvaiheessa. Luonnosvaiheessa tuotetta voidaan kutsua prototyypiksi (engl. \textit{prototype}) tai mallinnukseksi (engl. \textit{mockup}).
\item Keskivaiheessa tuote siirtyy tuotantoon ja valmistukseen. Tässä vaiheessa toteutetaan laadunvalvontaa ja kasaamista, ja voidaan puhua jo varsinaisesta tuotteesta. Valmis tuote siirtyy jakemluverkoston kautta itse asiakkaalle. Kun tuote on asiakkaalla, korostuu tuotteen käyttö sekä mahdollinen huolto ja asiakastuki.
\item Loppuvaiheessa tuotteen elinkaari päättyy. Tuotteen valmistusta ei koeta enää tarpeelliseksi, joten tässä vaiheessa huomio keskittyy tuotannon lopettamiseen ja tuotteen kierrätykseen.
\end{itemize}
Alemanni, ym. (2008) esittää PLM:n suorituskyvyn analysointia käsittelevässä artikkelissaan, että PLM-strategian keskittyessä olennaisesti tuotedatan hallintaan, on PLM-ohjelmisto olennainen osa strategian hyödyntämistä käytännössä.  \cite{alemanni_key_2008} Alemanni korostaa, että PLM-järjestelmän kehittäjän tulee kuitenkin toimia yhteistyössä asiakkaiden kanssa, jotta tuotteiden elinkaaren eri vaiheet ja prosessit voidaan implementoida osaksi ohjelmiston toimintoja siten. PLM-käsitteeseen liittyvien määritelmien lisäksi on tärkeää ymmärtää PLM-strategian ja -ohjelmistojen hyötyjä, jotta niiden hyödyntämisen motiivit voidaan ymmärtää. Tarkoituksena on siis vastata siihen, miksi ylipäätään PLM-järjestelmiä käytetään ja kehitetään.

\section{PLM-strategian hyödyt ja merkitys} \label{PLM strategian hyödyt}

PLM-strategian hyötyjä on käsitelty laajasti  \cite{alemanni_key_2008}  \cite{rivest_product_2012}. Strategian hyödyt voidaan jakaa kahteen osa-alueeseen: lyhyen ja pitkän aikavälin hyötyihin. Tietoteknisten PLM-järjestelmien tarkoituksena on taas mahdollistaa PLM-strategian käyttöön ottaminen ja hyödyntäminen käytännössä koko yrityksen tasolla. Näiden järjestelmien pääasiallisena tarkoituksena on koota tietoa yrityksen tuotteiden koko elinkaaren vaiheista keskitettyyn tietojärjestelmään. Tämä mahdollistaa laajojen tuotekantojen johdonmukaisen ja keskitetyn hallinnan yhteistyössä yrityksen eri osastojen ja kumppaneiden välillä.
Konkreettisia PLM:n hyötyjä voidaan havaita Lee, ym. (2008) toteuttamasta tutkimuksesta PLM:n hyödyntämisestä ilmailualalla: IBM-Dassaultin PLM-järjestelmää käytettiin lentokoneiden elinkaaren hallinnassa, jolloin ohjelmiston hyödyntäminen vähensi valmistusaikaa 16:sta kuukaudesta seitsemään kuukauteen. Lisäksi Teamcenter PLM -järjestelmä laski tuotantosykleihin käytettyä aikaa 35:llä prosentilla ja valmistusaikaa 66:lla prosentilla. Keskitetyssä järjestelmässä myös lentokoneiden huoltotarve voitiin ottaa paremmin huomioon jo suunnitteluvaiheessa, mikä suoraviivaisti myös tuotteiden huoltoa niiden elinkaaren aikana.  \cite{lee_product_2008}

\subsection{Lyhyen aikavälin hyödyt} \label{Lyhyen aikavälin hyödyt}

Lyhyellä aikavälillä PLM-strategia ja PLM-järjestelmän käyttöönotto voi vähentää aikaa jota käytetään työntekijöiden jokapäiväisten työtehtävien suorittamiseen. Strategian avulla yrityksen tuotetiedot ovat keskitetysti saatavilla, eikä ajantasaisia tietoa tarvitse kysellä eri osastojen välillä. Tämä johtaa siihen, että työntekijät voivat käyttää enemmän aikaa tehtäviin, jotka tuottavat yritykselle lisäarvoa arvoa. Lisäksi tuotteiden rakenteiden ymmärtäminen ja visualisointi helpottuu PLM-järjestelmän käyttöönoton myötä. Rakenteen ymmärrystä ja jaettavuutta eri osastojen välillä voidaan parantaa entisestään myös PLM-järjestemän raportoinnilla.  \cite{alemanni_key_2008}

\subsection{Pitkän aikavälin hyödyt} \label{Pitkän aikavälin hyödyt}

Pidemmällä aikavälillä hyödyt näkyvät konkreettisemmin PLM-strategiaa hyödyntävien yritysten tunnusluvuissa, erityisesti myyntikatteessa. PLM-järjestelmien keskeinen hyöty on prosessien suoraviivaistaminen, mikä johtaa usein tuotteiden saamiseen nopeammin ja useimmin markkinoille. Kun tuotteet pääsevät nopeammin suunnittelusta markkinoille, niiden suunnitteluun ja kehittämiseen käytetyt kustannukset laskevat.  \cite{bouhaddou_plm_2012}  \cite{alemanni_key_2008}

\section{Raportointi PLM-järjestelmässä} \label{Raportointi PLM-järjestelmässä}

Kuten osioissa \ref{PLM strategian hyödyt} todettiin, yksi PLM-järjestelmän hyödyistä on yksittäisten työntekijöiden työmäärän vähentäminen ja prosessien suoraviivaistaminen. Tietoteknisten järjestelmien etuna on varsinkin automoitu laskenta, joka voi vähentää inhimillisiä virheitä.  \cite{niu_organizational_2021}.  \cite{rakovic_digital_2022} Tätä automoitua laskentaa voidaan erityisesti hyödyntää raporttien muodostamisessa.

Koska PLM:n tarkoituksena mahdollistaa koko tuotantoketjun yhteistyön asiakkaiden, kehittäjien, toimittajien ja valmistajien välillä tuotteen eri elinkaaren vaiheissa,  \cite{bouhaddou_plm_2012} on tärkeää että tuotetieto elinkaaren eri vaiheissa on dokumentoitavissa, analysoitavissa ja helposti jaettavissa. Vaikka PLM-järjestelmien tietomallit ovat usein ohjelmistokohtaisia ja harvemmin standardinomaisia  \cite{SritiMohamed-Foued2012PTaS} on tärkeää, että tuotedataa on mahdollista viedä järjestelmän ulkopuolelle tallennettavaksi ja jaettavaksi.

PLM:n kontekstissa raporteilla tarkoitetaan tuotedataa kokoavia ja analysoivia kokonaisuuksia. Raportit voivat olla esimerkiksi PDF- tai Excel-tiedostoja, jotka kokoavat tuotetietoja ja suorittavat laskentaa visualisoimalla dataa esim. kuvaajin tai interaktiivisia "kojelautoja" (engl. \textit{dashboards}), jotka kokoavat useita kuvaajia ja laskettuja arvoja yksittäiseen käyttäjäystävälliseen näkymään. Tässä tutkielmassa keskitytään enemmän raporttitiedostojen tuottamiseen ohjelmallisesti, mutta usein näihin tiedostoihin on myös mahdollista upottaa kojelautamaisia ominaisuuksia, kuten kuvaajia ja tilastoja.

Raporttitiedostojen tuottamista varten monet PLM-järjestelmät tarjoavat "raportointimoottorin" osana PLM-ohjelmistoa, jonka tarkoituksena on kerätä analysoida dataa kokoamalla sitä dokumenttitiedostoformaatteihin. Alan standardina näistä formaateista raporttien kontekstissa lienee PDF-, Excel- ja HTML-pohjaiset raportit, sillä useimmat raportointimoottorit tarjoavat raportteja näissä tiedostomuodoissa ja ne ovat myös tuttuja suurimmalle osalle ohjelmiston käyttäjistä. \footnote{Ajatuksena oli tehdä raportoinnista oma osio ja tämän kappaleen yläotsikko, mutta aihealueena se on hyvin laaja ja väljä. Ehkä on parempi keskittyä raportointiin ja sen 
merkitykseen PLM-kontekstissa?}

\subsection{Osaluettelo PLM-järjestelmän sydämenä} \label{Osaluettelo PLM-järjestelmän sydämenä}

Yksi PLM-järjestelmän tärkeimmistä toiminnallisuuksista on tuotteen osaluettelon (BOM\nomenclature[BOM]{BOM}{Bill of Materials, osaluettelo, tuoterakenne})(engl. \textit{Bill of Materials}) esittäminen organisoidusti.  \cite{david_what_2016} Yksinkertaisuudessaan osaluettelo on lista kaikista osista, joita tarvitaan tuotteet valmistamiseen. Osaluettelossa jokainen yksittäiseen osaan voidaan liittää useita tietokenttiä kuten valmistaja, versio, materiaali ja määrä. Osaluettelo koostuu usein hierarkkisesti osakokoonpanoista, välikokoonpanoista, osakomponenteista ja yksittäisistä osista, eli se kerää dataa siitä, kuinka eri tuotteen komponentit ovat riippuvaisia toisistaan. Osaluetteloa voidaan käyttää viestintään esimerkiksi valmistuskumppanien välillä tai se voidaan rajoittaa yhteen tuotantoyksikköön.  \cite{jones_visualizing_2023}

Koska osaluettelot ovat hyvin olennainen osa PLM-järjestelmää, ovat ne tärkeä kohde myös raportoinnille. \cite{german_challenge_2016} Osaluetteloista tuotetut raportit voivat analysoida rakennetta pintaa syvemmältä sekä luoda helposti ymmärrettävän yleiskatsauksen massiivisen osaluettelon omaavaan tuotteeseen tarjoamalla samalla esimerkiksi graafeja ja statistiikkaa tuotteesta. Koska raportit voidaan tuottaa erillisinä sähköisinä dokumentteina, voidaan laskentaa jatkaa esimerkiksi Excel-raporttien tapauksessa, tai erityisesti PDF-raportit ovat omiaan arkistoinnille myöhempää käyttöä varten.

\subsection{Raportointi ja Business Intelligence} \label{Raportointi ja Business Intelligence}

Liiketoimintatiedon hyödyntämisellä (BI\nomenclature[BI]{BI}{engl. Business Intelligence, liiketoimintatiedon hyödyntäminen}), (engl. \textit{Business Intelligence}) tarkoitetaan yrityksen kykyä hyödyntää dataa merkityksellisellä tavalla. PLM:n kontekstissa BI korostuu etenkin tuotetiedon hyödyntämisessä. Tätä PLM:n ja BI:n yhteyttä on tutkinut Bosch-Mauchand, ym. (2014) artikkelissaan \textit{"Preliminary Requirements and Architecture Definition for Integration of PLM and Business Intelligence Systems"}  \cite{bayro-corrochano_preliminary_2014}. Bosch-Mauchand totetaa, että PLM järjestelmä kulkee käsikädessä BI:n kanssa ja PLM-järjestelmän tuotedatan integraatio ja sen jaettavuus eri järjestelmien välillä on hyvin tärkeää tuotetiedon merkityksellisen hyödyntämisen kannalta.
Bosch-Mauchand erittelee, että jotkin PLM-järjestelmät tarjoavat erillisiä moduuleja raporttien tuottamiseen, mutta harva raportointityökalu tai -moduuli hyödyntää BI:n periaatteita. Bosch-Mauchandin mukaan varsinkin kahden tyyppisillä raporteilla voidaan tuottaa lisäarvoa:  \cite{bayro-corrochano_preliminary_2014}
\begin{itemize}
\item Dokumenttien ja objektien määrällinen analysointi. Esimerkiksi näiden summien tai tyyppien laskenta.
\item PLM-järjestelmien ominaisuuksien käyttö. Esimerkiksi tuotteen osien uudelleenkäyttö ja tietokantakyselyt.
\end{itemize}
Näiden lisäksi raportointia voidaan hyödyntää IT-hallinnon osa-alueilla, kuten esimerkiksi järjestelmän suorituskyvyn monitoroinnin kannalta.

\section{Raportointimoottorit} \label{Raportointimoottorit}

Raportointimoottori on ohjelmisto tai osa ohjelmistokokonaisuutta, jolla voidaan luoda ohjelmallisesti raportteja. Tämä tarkoittaa sitä, että raportointimoottori saa vain syötedatan ja mahdollisen pohjan (engl. \textit{template}), mutta kaikki logiikka näiden yhdistämiseen sekä raportin koostamiseen jää raportointimoottorin vastuulle.  \cite{he_design_2010} Raportointimoottorilla ei tarkoiteta ainostaan ohjelmistokehittäjille käytössä olevaa sovellusta tai ohjelmistokirjastoa, sillä raportointimoottorit ovat usein konfiguroitavissa myös loppukäyttäjien toimesta. Täten raportointimoottoria kehittäessä tulee huomioida myös loppukäyttäjän tarpeet.  \cite{adhi_performance_2019}

Bambang Prasetya Adhi ym. vertailevat tutkimusartikkelissaan \textit{"Performance comparison of reporting engine birt, jasper report, and crystal report on the process business intelligence"} (2019)  \cite{adhi_performance_2019} suosituimpia kaupallisia (SAP Crystal Reports) ja avoimen lähdekoodin (BIRT ja Jasper Report) raportointimoottoreita. Ahdi ym. käyttävät kokeellisia menetelmiä mitatakseen kolmea raportointimoottorin osa-aluetta:
\begin{itemize}
\item Soveltuvuus, esimerkiksi kuinka hyvin raportointimoottori tukee erilaisen lähdedatan käyttöä
\item Käytettävyys, joka ilmenee oppimisen helppoutena sekä toiminnallisuuden loogisuutena ja tehokkaana käyttönä
\item Tehokkuus, joka mittaa itse järjestelmän tehokkuutta, esimerkiksi suoritusaikaa
\end{itemize}
Näitä osa-alueita arvioimalla voidaan tehdä perusteltuja päätöksiä raportointimoottorin valinnasta, joten nämä ovat tärkeitä seikkoja ottaa huomioon uuden raportointimoottorin suunnittelussa ja kehityksessä.

\subsection{Raportointimoottorin rakenne ja prosessi} \label{Raportointimoottorin rakenne ja prosessi}

Prosessina raportointimoottori toimii kolmella tasolla: data-, logiikka- ja esitystasolla. \cite{he_design_2010}

\subsubsection{Datataso}

Datatasolla reportointimoottori voi hakea dataa suoraan tietotokannasta tai esimerkiksi API\nomenclature[API]{API}{ohjelmointirajapinta, engl. Application Programming Interface}:n eli ohjelmointirajapinnan (engl. \textit{Application Programming Interface}) kautta. Raportointimoottorin tapauksessa ohjelmointirajapinta voi olla esimerkiksi hakurajapinta, jonka taustalla toimivan hakumoottorin avulla raportointimoottori voi hakea tarvitsemaansa dataa täsmällisemmin. Varsinkin PLM-järjestelmän kontekstissa hakumoottori on hyvin keskeinen osa PLM-järjestelmää. Datataso määrittelee siis mistä ja miten raportointimoottori hankkii lähtödataa sekä millaiset lähtötiedot sillä on koostaa raportti. \cite{he_design_2010} Näihin lähtötietoihin lukeutuu esimerkiksi mahdolliset raporttipohjat ja muut käyttäjän määrittämät asetukset.

\subsubsection{Logiikkataso}

Logiikkatasolla raportointimoottori jäsentelee lähtödataa ja suorittaa laskentaa. Lähtödatan formaatti on usein historiallisesti ollut XML\nomenclature[XML]{XML}{merkintäkielien standardi ja tiedostomuoto, engl. Extensible Markup Language} (engl. \textit{Extensible Markup Language}) sen ollessa yksi internetin yleisimmin käytetyistä dataformaateista, mutta JavaScriptin yleistyttyä myös JSON\nomenclature[JSON]{JSON}{engl. JavaScript Object Notation, yksinkertainen tiedostomuoto tiedon välitykseen ja tallennukseen} (engl. \textit{JavaScript Object Notation}) noussut suosituksi tiedostomuodoksi. JSONin etuna on sen suora integraatio JavaScriptin yhteyteen sekä sen nopeus verrattuna vanhempaan XML-standardiin. \cite{nurseitov_comparison_nodate} Logiikkatason toteuttaman laskennan avulla lähtödatasta voidaan luoda esimerkiksi graafeja ja sekä taulukoida dataa, mikä mahdollistaa lasketut sarakkeet ja summat. Tämä voidaan nähdä raportointimoottorin ytimenä, sillä se tuottaa merkityksellistä dataa usein vaikeaselkoisesta lähdedatasta. \cite{he_design_2010} Lisäksi logiikkatason tulee jäsentää data siten, että se on mahdollista kirjoittaa tiedostoon esitystasolla.

\subsubsection{Esitystaso}

Esitystasolla data kirjoitetaan tiedostoon, jolloin se voidaan tallentaa käyttäjäystävällisessä tiedostoformaatissa. \cite{he_design_2010} Suosituimpia tiedostoformaatteja ovat HTML-, Excel- ja PDF-tiedostoformaatit. \cite{he_design_2010} Esitystasolla raportin merkitys ilmenee käyttäjälle: lähtödata on esitetty helposti omaksuttavassa ja ymmärrettävässä muodossa, sekä raportti on luettavissa ja jaettavissa helposti yksittäisenä tiedostona.

\subsection{PLM-järjestelmä raportointimoottorin toimintaympäristönä} \label{PLM-järjestelmä raportointimoottorin toimintaympäristönä}

Rohleder ym. (2014) käsittelee tutkimuksessaan \textit{"Requirements Engineering in Business Analytics for Innovation and Product Lifecycle Management"} vaatimusten määrittelyä liiketoimintatiedon hyödyntämisessä PLM-järjestelmissä. Rohleder ym. korostaa, että PLM-järjestelmissä toimitaan usein Big datan eli \textit{massadatan} \nomenclature[Big data]{Big data}{suom. massadata, erittäin suuret ja järjestämättömät jatkuvasti lisääntyvät tietomassat} parissa, sillä heidän tutkimuksen mukaan yksittäisen auton tuoterakenne voi koostua noin 120 tuhannesta yksittäisestä osasta, joilla jokaisella on tyypillisesti omat CAD-mallit (tietokoneavusteisen suunniteluohjelman luomia tiedostoja)\nomenclature[CAD]{CAD}{engl. Computer Aided Design, tietokoneen käyttö suunnittelun apuvälineenä}, piirustukset ja metadataa. Lisäksi tuotteen useat versiot ja variantit nostavat lopullisen datan määrää eksponentiaalisesti. Lisäksi PLM-järjestelmiä käyttää tyyppillisesti eri työntekijöitä yrityksen eri osastoissa, jolloin tuonkulkujen poikkeavaisuus lisää PLM-järjelmien datan kompleksisuutta entisestään. Kompleksisuus johtaa usein siihen, että valmiit raportointimoottorit, varsinkin esimerkiksi taloudelliseen raportointiin erikoistuneet, eivät välttämättä sovi sellaisenaan käytettäväksi PLM:n kontekstissa. \cite{rohleder_requirements_2014}

Kuten kappaleessa \ref{Osaluettelo PLM-järjestelmän sydämenä} todettiin, osaluettelot ovat keskeiseen osa PLM-järjestelmään tallennetun datan esittämistä. Koska tuoteobjektit koostuvat osaluetteloista, myös PLM-järjestelmässä tuotteista koostettavat raportit perustuvat osaluetteloista kerättyyn lähtödataan. PLM-järjestelmän raportointimoottorit ovat siten erikoistuneita jäsentelemään ja kokoamaan hierarkkista dataa. \cite{rohleder_requirements_2014} PLM-järjestelmän tarjoamille raporteille on olennaista tuotteeseen ja sen kehitykseen liittyvät seikat, kuten esimerkiksi tuotteen osien toimittajien jakauma tai tuotteen muokkaushistoria. Lisäksi osaluettelon perusteella voidaan laskea yksittäisten osien summia rakenteessa tai esimerkiksi luoda raportteja tietyistä tuotteen osista, jotka täyttävät annetut kriteerit.

Mahdollisen raportointimoottorin ohella PLM-järjestelmät sisältävät usein hakumoottorin, jonka avulla voidaan etsiä tehokkaasti ja tarkasti tietokannasta annettujen kriteerien mukaisesti. Tiedon haku on yksi PLM-järjestelmän ydinominaisuuksista. \cite{enriquez_approach_2019} Raporttien muodostamisessa ulkoisen hakumoottorin hyödyntäminen vähentää itse raportointimoottorin kuormaa, jolloin raportointimoottorin toiminallisuuden kehittämisessä voidaan keskittyä enemmän laskentaan ja lisäarvon tuottamiseen. Täten PLM-järjestelmän tapauksessa lähtödatan hakeminen voi tapahtua haku-API:n välityksellä, jolloin raportteja voidaan muodostaa tuoterakenteiden lisäksi esimerkiksi jonkin tietyn hakulausekkeen perusteella.

PLM-järjestelmien käyttäjät ovat tyypillisesti suhteellisen suuren mittakaavan teollisuusyrityksiä. Useissa tapauksissa myös tuoterakenteet ovat valtavia \cite{rohleder_requirements_2014}, joten raportointimoottorin tulee olla tarpeeksi tehokas ja optimoitu, jotta myös suurista tietorakenteista on mahdollista koostaa raportteja siedettävässä suoritusajassa. Raportointimoottorin logiikkatason lisäksi PLM-järjestelmien käyttäjillä on myös tarpeita muokata raporttien ulkoasua raportointimoottorin esitystasolla. Esimerkiksi yrityksen logojen ja raportin visuaalisen ilmeen muokkaaminen on olennainen osa raportointimoottorin toiminnallisuutta.
