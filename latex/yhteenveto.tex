\chapter{Yhteenveto} \label{Yhteenveto}

\subsubsection{Tutkimuskysymykset}
\begin{itemize}
\item Mitä tulee ottaa huomioon raportointityökalua kehittäessä osaksi PLM-järjestelmää?
\item Millaisia ovat nykyiset raportointityökalut?
\item Miksi raportointityökaluja ja PLM-järjestelmiä kehitetään ja miten ne toimivat yhdessä?
\end{itemize}


Tämän tutkielman ensimmäisenä tutkimuskysymyksenä oli selvittää, mitä tulee ottaa huomioon kehittäessä raportointityökalua osaksi PLM-järjestelmää. Luvussa \ref{Raportointi ja PLM} tarkasteltiin PLM-järjestelmiä ja raportointia. Tutkielman motiiveja taustoitettiin perehtymällä PLM-strategian hyötyihin ja sen yhteyteen liiketoimintatiedon hyödyntämissä. Aiemman tutkimuksen perusteella voidaan todeta, että PLM-strategiassa liiketoimintatiedon hyödyntäminen korostuu etenkin tuotetiedon hyödyntämisenä. \cite{bayro-corrochano_preliminary_2014} PLM-strategia ja -järjestelmä -käsitteiden taustoittamisen jälkeen perehdyttiin raportointimoottoreihin ja työkaluihin luvussa \ref{Raportointimoottorit ja -työkalut}. 
