
\keywords{raportointi, PLM, raportointityökalu, ohjelmisto}
% TODO: good/bad keywords

\begin{abstract}
Raportoinnin tarkoituksena on kerätä, tallentaa ja esittää tietoa muodossa, joka on helposti ymmärrettävissä ja jaettavissa. Tehokkaan ja käyttäjäystävällisen raportoinnin mahdollistamiseksi on kehitetty tietoteknisiä raportointityökaluja, joiden avulla voidaan tuottaa raportteja annetun lähtödatan pohjalta.

Tässä tutkielmassa tarkastellaan raportointityökalun kehittämistä osaksi tuotteen elinkaaren hallintajärjestelemää. Tutkielmassa käsiteltävä elinkaaren hallintajärjestelmä on kaupallinen Sovelia PLM. Raportointityökalun kehittäminen koostuu raporttidokumenttien tuottamisesta ohjelmallisesti sekä raportointityökalun käyttöliittymän suunnittelusta ja kehittämisestä. Tarkastelemme PLM-järjestelmien ja raportointityökalujen yhteistoimintaa vedoten aiempaan kirjallisuuteen aiheesta, sekä tutkimme nykyisiä olemassa olevia raportointityökaluja. Aiemman kirjallisuuden pohjalta pyrimme ymmärtämään, millaista tuotteen elinkaaren hallintaohjelmistojen tuottama data on ja miten sen erityispiirteet vaikuttavat raportointiin. Tarkastelemalla olemassa olevia raportointityökaluja pyrimme muodostamaan raportointityökalujen tyypillisimmistä ominaisuuksista ja erityispiirteistä kokonaiskuvan, jonka perusteella pyrimme ymmärtämään, mitkä lähestymistavat toimivat Sovelia PLM:n tapauksessa ja mitkä eivät.

Tarkastelemalla aiempaa kirjallisuutta aiheesta havaitsimme, että PLM-data on massadatan kaltaista hierarkkista alati muuttuvaa dataa, mikä asettaa haasteita laadukkaalle raportoinnille. Olemassa olevia raportointityökaluja tarkastelemalla huomaisimme, että osa nykyisten raportointityökalujen lähestymistavoista olivat hyödyllisiä myös Sovelia PLM:n uuden raportointityökalun tapauksessa, mutta päädyimme osittain nykyaikaisempiin teknologiavalintoihin.
\end{abstract}
