\chapter{Case-tapaus: Raportointimoottorin kehittäminen osaksi Sovelia PLM-järjestelmää} \label{Case-tapaus: Raportointimoottorin kehittäminen osaksi Sovelia PLM-järjestelmää}

\section{Sovelia PLM} \label{PLM-järjestelmät}

Kehitettävän raportointityökalun toimintaympäristönä toimii kaupallinen PLM-järjestelmä, Sovelia PLM. Sovelia PLM:n kehitys on alkanut yli 30 vuotta sitten ja sen kehitys jatkuu aktiivisesti edelleen. Kuten muutkin PLM-järjestelmät, Sovelia PLM pyrkii ratkaisemaan valmistusalan yritysten haasteita liittyen tuotteen datan hallintaan sen koko elinkaaren ajan.\cite{soveliaAboutSovelia} Sovelia PLM:n erityispiirteenä on sisältämät valmiiksi konfiguroidut \textit{"templatet"} eli valmiit mallit objekteille ja prosesseille, jotka ovat muokattavissa asiakkaan tarpeiden mukaan. Lisäksi malleihin kuuluu muita valmiiksi konfiguroituja työkaluja että alalla hyväksi todettuja prosesseja.\cite{soveliaSoveliaGetting}

Sovelia PLM koostuu objekteista ja objektilinkeistä. Objekteja voidaan määritellä niiden attribuuttien avulla. Objektit voivat olla osia, piirrustuksia, dokumenttilinkkejä tai linkkejä muihin toimintohin.\cite{soveliaSoveliaGetting} Objektien linkkien avulla voidaan muodostaa osaluetteloita, jotka ova tärkeitä raportoinnin kohteita.

\subsection{Raportointi Sovelia PLM-järjestelmässä}

Sovelia PLM ja muut PLM-järjestelmät poikkeavat valmiiden raportointimoottoriratkaisujen tyypillisistä toimintaympäristöistä, sillä kuten luvussa \ref{PLM-järjestelmä raportointimoottorin toimintaympäristönä} huomattiin, PLM-data on usein kompleksista ja monimuotoista. Siemensin PLM-järjestelmän verkkosivuilla julkaistussa artikkelissa \textit{"The Challenge of Getting High Quality Reports out of PLM"}\cite{german_challenge_2016} (2016) esitellään haasteita, joita liittyy korkealaatuisten raporttien tuottamiseen PLM-järjestelmistä. Artikkeli nostaa esille osaluetteloiden (BOM) merkityksen PLM-järjestelmän raportoinnissa: on tärkeää että raporttien saama ja tuottama data on luotettavaa ja laadukasta, jotta päätöksenteko raporttien pohjalta olisi mahdollista. Osaluetteloihin voi tulla jopa satoja muutoksia päivittäin, joten raporttien tapauksessa on tärkeää, että niiden käyttäjät tietävät työskentelevänsä oikean datan kanssa. Täten raporttien ajantasaisuus ja selkeät aikaleimat ovat hyvin tärkeitä PLM-järjestelmän ja myös Sovelian tuottamille raportoille. Sovelia PLM:n raportointimoottorin etuna on hakurajapinnan käyttö, joka palauttaa aina hakupyynnön mukaisesti ajankohtaista dataa, mikä vähentää mahdollisten virheellisten tietojen määrää.

PLM-datan kompleksisuus tekee raportointityökalun kehittämisestä haasteellista, sillä myös Sovelia PLM vaatii erityisesti tarkoitusta varten kehitetyn raportointityökaluratkaisun. Tarkoitusta varten kehitetty ratkaisu on ohjelmistokehittäjälle työläs rakentaa alusta asti itse, mutta sen etuna on sen täysi muovautuvuus tarvetta varten. On kuitenkin todettava, että myös valmiin raportointimoottorin tai -työkalun implementoiminen Sovelia PLM:n toimintaympäristöön olisi todennäköistä työlästä juuri PLM-datan kompleksisuuden vuoksi. 

\subsubsection{Sovelia PLM:n vanha raportointityökalu}

Sovelia PLM -järjestelmässä on tuotannossa ja asiakkailla käytössä vanha raportointyökalu, joka tarjoaa raportteja PDF, Excel ja ZIP-tiedostoformaateissa. Raportteihin data kerätään \textit{"rakenneagentin"} avulla, joka on nimensä mukaisesti osaa kulkea objekti-linkki-suhteita pitkin ja täten kerätä tarvittavan lähdedatan raportin koostamista varten. Itse raporttitiedostoon kirjoittaminen tapahtuu Java-ohjelmointikielen Apache FOP-ohjelmakirjaston\cite{noauthor_apachetm_nodate} avulla XSL-\nomenclature[XSL]{XSL}{XML-kieliperhe, joka mahdollistaa XML-pohjaisten tiedostojen ulkoasun ja rakennemuutoksen määrittelyn} (engl. \textit{Extensible Stylesheet Language}, kieliperhe, joka mahdollistaa XML-pohjaisten tiedostojen ulkoasun ja rakennemuutoksen määrittelyn) ja XML-tietorakenteisiin tallennetun datan antaman ohjeistuksen avulla.

Vaikka tämä ratkaisu on toimiva ja edelleen kelvollinen tekniikka raporttien generointiin ohjelmallisesti ennalta määriteltyn datan mukaisesti, se kohtaa ongelmia datamäärien lisääntyessä. Näitä ongelmia ovat mm. liiallinen muistinkäyttö palvelimilla, prosessin hitaus varsinkin suurempien tuoterakenteiden tapauksissa, konfiguroinnin haasteellisuus ja toiminnon ylläpitö muuttuvassa toimintaympäristössä.

\subsubsection{Uuden raportointyökalun haasteet}

\section{Nykyiset raportointimoottorit}

\textit{Alustus ja esittely nykyisiin raportointityökaluihin, niiden ominaisuuksiin ja käytettyihin tekniikoihin. Mitä voimme oppia näistä ratkaisuista, mitä tekniikoita nämä käyttävät ja millaisia lopputulemia ne tarjoavat?}
