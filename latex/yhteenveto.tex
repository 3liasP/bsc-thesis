\chapter{Yhteenveto} \label{Yhteenveto}

Tämän tutkielman ensimmäisenä tutkimuskysymyksenä oli selvittää, miten PLM-järjestelmät ja raportointityökalut toimivat yhdessä. Kysymystä taustoitettiin luvussa \ref{Raportointi ja PLM} tarkastelemalla PLM-järjestelmiä ja raportointia erikseen sekä niiden yhteyttä. Tutkielman motiiveja taustoitettiin perehtymällä PLM-strategian hyötyihin ja PLM:n yhteyteen liiketoimintatiedon hyödyntämisessä. Aiempaan kirjallisuuteen vedoten todettiin, että PLM-järjestelmä olennainen osa PLM-strategian hyödyntämistä käytännössä.\cite{alemanni_key_2008} Osaluettelon (engl. \textit{Bill of Materials, BOM}) merkitystä korostettiin yhtenä PLM-järjestelmän tärkeimmistä toiminnallisuuksista\cite{david_what_2016} luvussa \ref{Osaluettelo PLM-järjestelmässä}. Samassa luvussa perusteltiin, että osaluettelot ovat tärkeä osa PLM-järjestelmän raportointia \cite{german_challenge_2016}. Aiemman tutkimuksen perusteella voidaan myös todeta, että PLM-strategiassa liiketoimintatiedon hyödyntäminen korostuu etenkin osaluetteloista koostuvan tuotetiedon hyödyntämisenä. \cite{bayro-corrochano_preliminary_2014}  PLM-strategian ja -järjestelmän taustoittamisen jälkeen perehdyttiin raportointimoottoreihin ja työkaluihin luvussa \ref{Raportointimoottorit ja -työkalut}. Aluksi määriteltiin raportointityökalun ja -moottorin käsitteet, jonka jälkeen tarkasteltiin raportointimoottorin rakennetta ja prosessia. Luvussa \ref{Raportointimoottorin rakenne ja prosessi} raportointimoottori jaettiin aiemman tutkimuksen tavan mukaisesti kolmelle tasolle: data, -logiikka ja esitystasolle. \cite{he_design_2010}.

Raportointimoottorin tasojen tarkastelun jälkeen pyrittiin vastaamaan ensimmäiseen tutkimuskysymykseen PLM-järjestelmien ja raportointityökalujen yhteydestä. Tarkastelemalla ensin raportointia PLM-järjestelmässä todettiin, että tässä tutkielmassa keskitytään erityisesti digitaalisten raporttidokumenttien tuottamiseen ohjelmallisesti. Nämä raporttidokumentit ovat esimerkiksi PDF- tai Excel-muotoisia tiedostoja. PLM-järjestelmän todettiin olevan haasteellinen ja yksilöllinen toimintaympäristö aiemman kirjallisuuden perusteella, sillä valmiit raportointimoottorit eivät välttämättä sovi sellaisenaan käytettäväksi usein massadatan kaltaisen PLM-datan kanssa. \cite{rohleder_requirements_2014} Lisäksi huomattiin, että PLM-järjestelmän raportointimoottorit ovat erikoistuneita toimimaan hierarkkisen datan parissa. \cite{rohleder_requirements_2014} Kehittääkseen raportointityökalua PLM-järjestelmään tulee ymmärtää PLM-järjestelmä toimintaympäristönä, johon liittyy erityisesti hierarkkinen massadata, raporttidokumenttien tiedostomuodot ja muiden PLM-järjestelmän ominaisuuksien hyödyntäminen. Muiden ominaisuuksien hyödyntämisellä tarkoitetaan erityisesti hakurajapinnan käyttöä raportointimoottorin datalähteenä, sillä hakutoiminto on yksi PLM-järjestelmän keskeisimmistä toiminnallisuuksista. \cite{enriquez_approach_2019} Merkittävänä haasteena raporttien tuottamiselle PLM-järjestelmässä pidettiin raporttien tuottaman tiedon ajankohtaisuutta ja luotettavuutta alati muuttuvassa sekä laajassa PLM-datavirrassa. \cite{german_challenge_2016} 

Aiemman kirjallisuuden tarkastelu avasi monia näkökulmia PLM-järjestelmien ja raportoinnin yhteistoimintaan, joita päästiin soveltamaan käytännössä raportointityökalun kehitysprosessissa. Käymällä läpi alan tutkimuksia ja asiantuntija-artikkeleita, oli mahdollista hahmottaa selkeämpiä yhteyksiä siitä, miten tuotteen elinkaaren hallintaohjelmistot ja raportointityökalut liittyvät toisiinsa. Uuden raportointityökalun kehittämisessä korostui erityisesti tarve  hierarkkisen PLM-datan perusteelliselle ymmärrykselle ja sen vaikutuksiin raporttien luomisessa. Hierarkkisen rakenteen ymmärtäminen oli tärkeää, kun pyritään vastaamaan kysymyksiin siitä, miten eri tuotteen tasojen tiedot liittyvät toisiinsa ja miten raportit voivat tarjota kokonaisvaltaisen näkymän tuotteesta. Hierakkinen data korostui alati kirjallisuutta tutkiessa ja merkitys myös Sovelia PLM:n tapauksessa oli ilmeinen. On suositeltavaa käyttää räätälöityjä ratkaisuja PLM-järjestelmän raportointityökalun saaman hierarkkisen datan käsittelyssä, esimerkiksi ohjelmiston sisäisenä tietomallina, jotta erityislaatuinen rakenteellinen data voidaan käsitellä skaalautuvasti tuoterakenteiden kasvaessa.

Luvussa \ref{Tapaus: Sovelia PLM:n raportointityökalu} syvennyttiin valmiiden raportointityökaluohjelmistojen ominaisuuksien tarkastelemiseen. Tarkastelemalla nykyisiä raportointityökaluja pyrittiin vastaamaan siihen, millaisia nykyiset raportointityökalut ovat ja millaisia ominaisuuksia ne tarjoavat. Sovelia PLM esiteltiin toimintaympäristönä ja sen erityispiirteitä sekä käsitteitä avattiin, jotta kehitettävän raportointityökalun toimintaympäristöä voitaisiin ymmärtää paremmin. Tutustuimme Sovelia PLM:n vanhaan raportointimoottoriin esittelemällä sen käyttämiä teknologioita ja niihin liittyviä haasteita ja ongelmia.

Vanhan Sovelia PLM:n raportointityökalun esittelyn jälkeen tarkasteltiin luvussa \ref{Nykyisten yleiskuva} nykyisten olemassa olevien raportointityökalujen ominaisuuksia erottelemalla niiden käyttämiä lisenssejä, datalähteitä ja ohjelmointikieliä. Tutustuimme lyhyesti myös näiden raportointityökalujen konfigurointimahdollisuuksiin sekä asennustapaan. Jotta ymmärtäisimme näiden työkalujen tuottamia raportteja, analysoitiin myös raporttien tiedostomuotoja ja niiden valintojen motiiveja. Nykyisten raportointityökalujen tarkastelun jälkeen vertasimme näistä saatuja havaintoja uuden raportointityökalun kehitysprosessiin ja teknologiavalintoihin. Huomasimme, että osa nykyisten raportointityökalujen lähestymistavoista olivat hyödyllisiä myös Sovelia PLM:n uuden raportointityökalun tapauksessa, mutta päädyimme osittain nykyaikaisempiin ratkaisuihin. Esimerkiksi tarkasteltujen raportointityökalujen keskuudessa Java oli ylivoimaisesti suosituin ohjelmointikieli, mutta päädyimme modernimman Node.js palvelinympäristön valintaan. Ohjelmistokehittäjälle Node.js on mielestäni käyttäjäystävällinen ja erittäin runsaasti dokumentoitu alusta palvelinympäristölle ja skaalautuvuutensa sekä runsaasti saatavilla olevien hyödyllisten ohjelmakirjastojen vuoksi se hyvä valinta myös raportointimoottorille. Nykyisten raportointityökalujen suosimaa JSONia käytettiin taas myös uuden raportointityökalun lähtödatana, sekä uuden raportointityökalun tarjoamat ulostuloformaatit vastasivat läheisesti nykyisten tarkasteltujen raportointityökalujen suosimia tiedostomuotoja.

Tutkielman tulokset ovat linjassa aiemman kirjallisuuden kanssa, mutta vertailu olemassa oleviin raportointimoottoreihin ei ollut yhtä suoraviivaista. Kuten luvussa ref{Nykyisten yleiskuva} kerrottiin, tarkasteltujen raportointityökalujen kehittäminen oli aloitettu lähes kaikkien kohdalla 2000-luvun alkupuolella, joten näiden työkalujen teknologiaratkaisut olivat ajan mukaisia. Kokonaan uutta työkalua kehittäessä on kuitenkin huomioitava nykyaikaiset trendit ja saatavilla olevat teknologiat, joten on tärkeää, että raportointityökalun kehittäjä ottaa huomioon myös nykyaikaiset teknologiavaihtoehdot. Vertailemalla raportointityökaluja saatiin kuitenkin selkeä yleiskuva niiden ominaisuuksista, joka ehdottomasti suoraviivaisti Sovelia PLM:n raportointityökalun kehitysprosessia.

Tässä tutkielmassa ei otettu kuitenkaan huomioon sitä, että markkinoilla on useita raportointityökaluja, joita tässä ei tarkasteltu. Tarkasteltavat raportointityökalut valittiin niiden yleisen suosion sekä vertailukelpoisuuden perustella kehitettävään raportointimoottorin verrattuna, joten tutkielman tuloksia ei voida käyttää yleiskuvana kaikista saatavilla olevista raportointityökaluista. Lisäksi tarkastelimme valituista raportointityökaluista vain muutamaa ominaisuutta, emmekä syventyneet näiden työkalujen toteutusten teknisiin seikkoihin. Kehitettävä raportointityökalu on myös erillinen yksittäistapaus, sillä se on kehitetty osaksi yhtä kaupallista PLM-järjestelmää. Täten muut PLM-järjestelmät voivat poiketa siitä suuresti eikä tässä tutkielmassa kuvattua implementaatiota voida välttämättä käyttää sellaisenaan.

Jatkotutkimuksena aiheesta voitaisiin perehtyä PLM-järjestelmän ja siten raportoinnin sisäisiin tietomalleihin, tarkastella vaihtoehtoja raporttien sisällön ja ulkoasun muokkaamiselle sekä syventyä raportointityökalun elinkaareen. Käytännössä tutkielman tuloksia voidaan hyödyntää dokumentoituna esimerkkinä raportointityökalun kehittämisestä PLM-järjestelmään, jolloin tämän tutkielman havaintoja voidaan hyödyntää vastaavanlaisia järjestelmiä kehittäessä esimerkiksi teknologiavalintojen ja haasteiden ymmärtämiseksi. Tutkielman perustella voidaankin todeta, että raportointityökalun kehittäminen PLM-järjestelmään vaatii hierarkkisen massadatan kaltaisen PLM-datan tuntemista sekä ymmärrystä raportointityökalujen ominaisuuksien heikkouksista ja vahvuuksista.

