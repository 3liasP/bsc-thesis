
\keywords{raportointi, PLM, raportoinityökalu, ohjelmisto}
% TODO: good/bad keywords

\begin{abstract}
Raportoinnin tarkoituksena on kerätä, tallentaa ja esittää tietoa muodossa, joka on helposti ymmärrettävissä ja jaettavissa. Tehokkaan ja käyttäjäystävällisen raportoinnin mahdollistamiseksi on kehitetty raportointityökaluja, joiden avulla voidaan voidaan luoda raportteja saadun lähtödatan pohjalta.

Tässä tutkielmassa tarkastellaan raportointityökalun kehittämistä osaksi kaupallista tuotteen elinkaaren hallintajärjestelemää, Sovelia PLM:ää. Raportointityökalun kehittäminen koostuu raporttidokumenttien tuottamisesta ohjelmallisesti sekä raportointityökalun käyttöliittymän suunnittelusta ja kehittämisestä. Tarkastelemme PLM-järjestelmien ja raportointityökalujen yhteistoimintaa vedoten aiempaan kirjallisuuteen aiheesta, sekä tutkimme nykyisiä olemassa olevia raportointityökaluja. Aiemman kirjallisuuden pohjalta pyrimme ymmärtämään, millaista tuotteen elinkaaren hallintaohjelmistojen tuottama data on ja miten sen erityispiirteet vaikuttavat raportointiin. Havaitsimme erityisesti, että PLM-data on massadatan kaltaista hierarkkista alati muuttuvaa dataa, mikä asettaa haasteita laadukkaalle raportoinnille. 

Tarkastelemalla olemassa olevia raportointityökaluja pyrimme muodostamaan raportointityökalujen tyypillisimmistä ominaisuuksista ja erityispiirteistä kokonaiskuvan, jonka perusteella pyrimme ymmärtämään, mitkä lähestymistavat toimivat Sovelia PLM:n tapauksessa ja mitkä eivät. Huomasimme, että osa nykyisten raportointityökalujen lähestymistavoista olivat hyödyllisiä myös Sovelia PLM:n uuden raportointityökalun tapauksessa, mutta päädyimme osittain nykyaikaisempiin teknologiavalintoihin.

\end{abstract}
