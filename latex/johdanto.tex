\chapter{Johdanto} \label{Johdanto}

Tietotekniikan avulla voidaan tehostaa ja helpottaa työnteon tuottavuutta, kun samaan tehtävään käytetty aika vähenee. Tietotekniikan hyödyntäminen raportointiin hyvin luonnollista, sillä raportit ovat useimmiten digitaalisesti tuotettuja dokumentteja, joiden kokoaminen vaatii jonkinlaista laskentaa. Raportointidatan kerääminen ja jäsenteleminen manuaalisesti on hyvin vaivalloista ja hidasta, mikä vuoksi tietojärjestelmät voivat tarjota raportointityökaluja, joiden tarkoituksena on koota raportti automoidusti määritellystä lähdedatasta.

Raportoinnin ydinajatuksena on tuottaa tietoa muodossa, joka on helposti ymmärrettävissä ja jaettavissa. Raportointityökalujen avulla olemassa olevasta suuresta määrästä dataa voidaan tuottaa selkeä ja jäsennelty esitys, joka kokoaa lähdedatan tärkeimmät seikat helposti yhdellä silmäyksellä omaksuttavaan muotoon.

Tämän työn tarkoituksena oli toteuttaa raportointityökalu osaksi Sovelia PLM -järjestelmää. Sovelia PLM on kaupallinen tuotteen elinkaaren hallintajärjestelmä(PLM\nomenclature[PLM]{PLM}{engl. Product Lifecycle Management, tuotteen elinkaaren hallinta}), (engl. \textit{Product Lifecycle Management}). PLM-järjestelmän pääasiallisena tarkoituksena on koota tietoa yrityksen tuotteiden koko elinkaaren vaiheista keskitettyyn tietojärjestelmään. Tämä keskitetty tietojärjestelmää on käytettävissä yrityksen eri työryhmien ja liiketoimintajärjestelmien välillä, minkä tarkoituksena on vähentää virheellisten tuotetietojen aiheuttamia turhia kustannuksia sekä viivästyksiä ja siten nopeuttaa yrityksen prosessia saada kehitetty tuote markkinoille.

Raportointityökalu voidaan nähdä yhtenä PLM järjestelmälle lisäarvoa tuottavana ominaisuutena. Luotettavan, tehokkaan ja mukautuvan raportointityökalun avulla PLM-järjestelmä voi tuottaa enemmän lisäarvoa sen käyttäjille tarjoamalla mahdollisuuden jakaa, tallentaa ja analysoida tuotedataa eri tiedostoformaateissa sekä yrityksen sisäisten työryhmien että ulkoisten toimijoiden välillä. PLM-järjestelmiä käyttävillä yrityksillä on tyypillisesti suuria määriä tuotetietoja ja syviä tuoterakenteita, jolloin myös raportoinnin suorituskykyvaatimukset korostuvat.

PLM-järjestelmien tietomallit voidaan jakaa dokumentti- ja relaatiodatapohjaisiin tietorakenteisiin. \cite{david_what_2016} Koska Sovelia PLM -järjestelmä perustuu relaatiodatapohjaiseen tietomalliin, myös tässä tutkielmassa käsitellään raportointia nimenomaan relaatiodatan pohjalta.

Raportointityökalu integroituu osaksi Sovelian nykyistä lähdekoodia ja sen palvelinkomponentteja. Ohjelmakokonaisuus koostu palvelinkomponentista, joka tuottaa raporttitiedoston raportoinnin kohteena olevasta objektista, sekä konfigurointityökalusta, jonka avulla pääkäyttäjä voi muokata raporttien ulkonäköä ja rakennetta.

\subsubsection{Tutkimuskysymykset}
\begin{itemize}
\item Mitä tulee ottaa huomioon raportointityökalua kehittäessä osaksi PLM-järjestelmää?
\item Millaisia ovat nykyiset raportointityökalut?
\item Miksi raportointityökaluja ja PLM-järjestelmiä kehitetään ja miten ne toimivat yhdessä?
\end{itemize}

\subsubsection{Tutkimusmenetelmät}
\begin{itemize}
\item Kirjallisuusanalyysi
\item Kvalitatiivinen case-analyysi
\end{itemize}

\footnote{Johdannon sisältöä tulee arvioida uudelleen, kun tutkielman sisältö on valmiimpi. Tutkielman rakenne tulee esittää tiivistetysti ja nykyistä tekstiä tulee karsia.}
