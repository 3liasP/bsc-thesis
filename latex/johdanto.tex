\chapter{Johdanto} \label{Johdanto}

Raportoinnin ydinajatuksena on tuottaa tietoa muodossa, joka on helposti ymmärrettävissä ja jaettavissa. Raportointityökalujen avulla olemassa olevasta suuresta määrästä dataa voidaan tuottaa selkeä ja jäsennelty esitys, joka kokoaa lähdedatan tärkeimmät seikat helposti yhdellä silmäyksellä omaksuttavaan muotoon.

Tietotekniikan avulla voidaan tehostaa ja helpottaa työnteon tuottavuutta, kun samaan tehtävään käytetty aika vähenee. Tietotekniikan hyödyntäminen raportointiin hyvin luonnollista, sillä raportit ovat useimmiten digitaalisesti tuotettuja dokumentteja. Raportointidatan kerääminen ja jäsenteleminen manuaalisesti on hyvin vaivalloista ja hidasta, joten siksi useat tietojärjestelmät tarjoavat raportityökalun, joka kokoaa raportin automoidusti määritellystä lähdedatasta.

Tämän työn tarkoituksena oli toteuttaa raportointityökalu osaksi Sovelia PLM -järjestelmää. Sovelia PLM on kaupallinen tuotteen elinkaaren hallintajärjestelmä(PLM\nomenclature[PLM]{PLM}{engl. Production Lifecycle Management, tuotteen elinkaaren hallinta}), jonka pääasiallisena tarkoituksena on koota tietoa yrityksen tuotteiden koko elinkaaren vaiheista keskitettyyn tietojärjestelmään. Tämä keskitetty tietojärjestelmää on käytettävissä yrityksen eri työryhmien ja liiketoimintajärjestelmien välillä, minkä tarkoituksena on vähentää virheellisten tuotetietojen aiheuttamia turhia kustannuksia sekä viivästyksiä ja siten nopeuttaa yrityksen prosessia saada kehitetty tuote markkinoille.

Raportointityökalu voidaan nähdä yhtenä PLM ydinominaisuuksista. Luotettavan, tehokkaan ja mukautuvan raportointityökalun avulla PLM-järjestelmä voi tuottaa enemmän lisäarvoa sen käyttäjille tarjoamalla mahdollisuuden jakaa, tallentaa ja analysoida tuotedataa eri tiedostoformaateissa sekä yrityksen sisäisten työryhmien että ulkoisten toimijien välillä. PLM-järjestelmiä käyttävillä yrityksillä on tyypillisesti suuria määriä tuotetietoja ja syviä tuoterakenteita, jolloin myös raportoinnin suoritykykyvaatimukset korostuvat.

PLM-järjestelmien tietomallit voidaan jakaa dokumentti- ja relaatiodata-pohjaisiin tietorakenteisiin. \cite{david_what_2016} Koska Sovelia PLM -järjestelmä perustuu relaatiodata-pohjaiseen tietomalliin, myös tässä tutkielmassa käsitellän raportointia nimenomaan relaatiodatan pohjalta.

Raportointityökalu integroituu osaksi Sovelian nykyistä lähdekoodia ja sen palvelinkomponentteja. Ohjelmakokonaisuus koostu palvelinkomponentista, joka tuottaa raporttitiedoston raportoinnin kohteena olevasta objektista, sekä konfigurointityökalusta, jonka avulla pääkäyttäjä voi muokata raporttien ulkonäköä ja rakennetta.
