\chapter{Raportointi ja PLM-järjestelmät} \label{Raportointi ja PLM}

\section{PLM-strategia ja PLM-järjestelmät} \label{PLM-strategia ja PLM-järjestelmät}

Tuotteen elinkaaren hallinta eli PLM voidaan nähdä yrityksen strategiana hallita tuotetietoja. PLM strategiana koostuu tuotteista, organisaatioista, työmenetelmistä, prosesseista, ihmisistä ja lopulta usein myös tietoteknisestä elinkaaren hallintajärjestelmästä. Tietoteknisen elinkaaren hallintajärjestelmän, eli PLM-järjestelmän, tarkoituksena on mahdollistaa ja helpottaa PLM-strategian käyttöönottoa ja hyödyntämistä käytännössä. PLM-järjestelmä sisältää siis erilaisia toiminnallisuuksia, joiden avulla sen käyttäjät voivat hallita tuotetietoja keskitetyssä järjestelmässä tuotteen koko elinkaaren ajan.

Tuotteen elinkaari voidaan jakaa alku-, keski- ja loppuvaiheeseen. Tuotteen elinkaaren pääpiirteet on hyvä ymmärtää, jotta PLM-käsitettä voidaan tarkastella syvällisemmin. Bouhaddou \cite{bouhaddou_plm_2012} \textit{"PLM Model for Supply Chain Optimization"} määrittelee tuotteen elinkaaren vaiheet ja niiden piirteet seuraavasti:
\begin{itemize}
\item Alkuvaiheessa tuotteen vaatimuksia määritellään ja tuote on luonnosvaiheessa. Luonnosvaiheessa tuotetta voidaan kutsua prototyypiksi (engl. \textit{prototype}) tai mallinnukseksi (engl. \textit{mockup}).
\item Keskivaiheessa tuote siirtyy tuotantoon ja valmistukseen. Tässä vaiheessa toteutetaan laadunvalvontaa ja kasaamista ja voidaan puhua jo varsinaisesta tuotteesta. Valmis tuote siirtyy jakeluverkoston kautta itse asiakkaalle. Kun tuote on asiakkaalla, korostuu tuotteen käyttö sekä mahdollinen huolto ja asiakastuki.
\item Loppuvaiheessa tuotteen elinkaari päättyy. Tuotteen valmistusta ei koeta enää tarpeelliseksi, joten tässä vaiheessa huomio keskittyy tuotannon lopettamiseen ja tuotteen kierrätykseen. \cite{bouhaddou_plm_2012}
\end{itemize}
Alemanni ym. \cite{alemanni_key_2008} esittävät artikkelissaan \textit{"Key performance indicators for PLM benefits evaluation: The Alcatel Alenia Space case study”}, että PLM-strategian keskittyessä olennaisesti tuotedatan hallintaan on PLM-ohjelmisto eli PLM-järjestelmä olennainen osa strategian hyödyntämistä käytännössä. Alemanni korostaa, että PLM-järjestelmän kehittäjän tulee kuitenkin toimia yhteistyössä asiakkaiden kanssa, jotta tuotteiden elinkaaren eri vaiheet ja prosessit voidaan sisällyttää osaksi ohjelmiston toimintoja asiakaskohtaisesti. PLM-käsitteeseen liittyvien määritelmien lisäksi on tärkeää ymmärtää PLM-strategian ja -ohjelmistojen hyötyjä, jotta niiden hyödyntämisen motiivit voidaan ymmärtää. Tarkoituksena on vastata siihen, miksi ylipäätään PLM-järjestelmiä käytetään ja kehitetään.

\section{PLM-strategian hyödyt ja merkitys} \label{PLM-strategian hyödyt ja merkitys}


\textbf{Lyhyellä aikavälillä} PLM-strategian ja PLM-järjestelmän käyttöönotto voivat vähentää aikaa, jota käytetään työntekijöiden jokapäiväisten työtehtävien suorittamiseen. Strategian avulla yrityksen tuotetiedot ovat keskitetysti saatavilla eikä ajantasaista tietoa tarvitse kysellä eri osastojen välillä. Tämä johtaa siihen, että työntekijät voivat käyttää enemmän aikaa tehtäviin, jotka tuottavat yritykselle lisäarvoa. Lisäksi tuotteiden rakenteiden ymmärtäminen ja visualisointi helpottuu PLM-järjestelmän käyttöönoton myötä. Tuoterakenteiden ymmärrystä ja jaettavuutta eri osastojen välillä voidaan parantaa entisestään myös PLM-järjestemän raportointiominaisuuksilla.  \cite{alemanni_key_2008}

\textbf{Pitkällä aikavälillä} hyödyt alkavat näkyä konkreettisemmin PLM-strategiaa hyödyntävien yritysten tunnusluvuissa, erityisesti myyntikatteessa. PLM-järjestelmien keskeinen hyöty on prosessien suoraviivaistaminen, mikä johtaa usein tuotteiden saamiseen markkinoille nopeammin ja useammin. Kun tuotteet pääsevät nopeammin suunnittelusta markkinoille, niiden suunnitteluun ja kehittämiseen käytetyt kustannukset luonnollisesti laskevat.  \cite{bouhaddou_plm_2012}  \cite{alemanni_key_2008}

\subsection{Osaluettelo PLM-järjestelmässä} \label{Osaluettelo PLM-järjestelmässä}

Yksi PLM-järjestelmän tärkeimmistä toiminnallisuuksista on tuotteen osaluettelon (engl. \textit{Bill of Materials, BOM}) esittäminen organisoidusti.\cite{david_what_2016} Yksinkertaisuudessaan osaluettelo on lista kaikista osista, joita tarvitaan tuotteet valmistamiseen. Osaluettelossa jokaiseen yksittäiseen osaan voidaan liittää useita tietokenttiä, kuten tuotteen valmistaja, versio, materiaali ja määrä. Tuoterakenne on osaluettelo, joka koostuu hierarkkisesti osakokoonpanoista, välikokoonpanoista, osakomponenteista ja yksittäisistä osista, eli se kerää dataa siitä, kuinka eri tuotteen komponentit ovat riippuvaisia toisistaan. Osaluetteloa voidaan käyttää viestintään, esimerkiksi valmistuskumppanien välillä, tai se voidaan rajoittaa yhteen tuotantoyksikköön.  \cite{jones_visualizing_2023}

Koska osaluettelot ovat hyvin olennainen osa PLM-järjestelmää, ovat ne tärkeä kohde myös raportoinnille. \cite{german_challenge_2016} Osaluetteloista tuotetut raportit voivat analysoida rakennetta pintaa syvemmältä sekä luoda helposti ymmärrettävän yleiskatsauksen massiivisen osaluettelon omaavaan tuotteeseen tarjoamalla samalla esimerkiksi graafeja ja statistiikkaa tuotteesta. Koska raportit voidaan tuottaa erillisinä sähköisinä dokumentteina, voidaan laskentaa jatkaa esimerkiksi Excel-raporttien tapauksessa, tai erityisesti PDF-raportit ovat omiaan arkistoinnille myöhempää käyttöä varten.

\subsection{Raportointi ja Business Intelligence} \label{Raportointi ja Business Intelligence}

Liiketoimintatiedon hyödyntämisellä , (engl. \textit{Business Intelligence}, BI\nomenclature[BI]{BI}{engl. Business Intelligence, liiketoimintatiedon hyödyntäminen}) tarkoitetaan yrityksen kykyä hyödyntää dataa merkityksellisellä tavalla. PLM:n kontekstissa BI korostuu etenkin tuotetiedon hyödyntämisessä. Tätä PLM:n ja BI:n yhteyttä on tutkinut Bosch-Mauchand ym. \cite{bayro-corrochano_preliminary_2014} \textit{"Preliminary Requirements and Architecture Definition for Integration of PLM and Business Intelligence Systems"}. Bosch-Mauchand totetaa, että PLM järjestelmä kulkee käsikädessä BI:n kanssa ja PLM-järjestelmän tuotedatan integraatio ja sen jaettavuus eri järjestelmien välillä on hyvin tärkeää tuotetiedon merkityksellisen hyödyntämisen kannalta.
Bosch-Mauchand erittelee, että jotkin PLM-järjestelmät tarjoavat erillisiä moduuleja raporttien tuottamiseen, mutta harva raportointityökalu tai -moduuli hyödyntää BI:n periaatteita. Bosch-Mauchandin mukaan varsinkin kahden tyyppisillä raporteilla voidaan tuottaa lisäarvoa \cite{bayro-corrochano_preliminary_2014}:
\begin{itemize}
\item Dokumenttien ja objektien määrällinen analysointi. Esimerkiksi näiden summien tai tyyppien laskenta.
\item PLM-järjestelmien ominaisuuksien käyttö. Esimerkiksi tuotteen osien uudelleenkäyttö ja tietokantakyselyt. \cite{bayro-corrochano_preliminary_2014}
\end{itemize}
Näiden lisäksi raportointia voidaan hyödyntää IT-hallinnon osa-alueilla, kuten esimerkiksi järjestelmän suorituskyvyn monitoroinnissa.

\section{Raportointimoottorit ja -työkalut} \label{Raportointimoottorit ja -työkalut}

Raportointimoottorit ja -työkalut ovat molemmat tietojen käsittelyyn ja analysointiin tarkoitettuja ohjelmistoja. Tämän tutkielman kontekstissa nämä käsitteet määritellään seuraavanlaisesti: Raportointimoottorit ovat ohjelmistoja, jotka automatisoivat raporttien luomisen ohjelmallisesti. Tyypillisesti raportointimoottori saa syötedatan, jonkinlaisen ohjeistuksen tai raporttipohjan (engl. \textit{report template}) ja kaikki logiikka näiden yhdistämiseen sekä lopullisen raportin muodostamiseen jää raportointimoottorin vastuulle.  \cite{he_design_2010} Raportointityökalu on taas laajempi termi, ja sillä tarkoitetaan ohjelmistoa, joka on suunniteltu auttamaan raporttien luomisessa, hallinnassa ja jakelussa. Yksinkertaistettuna raportointityökalut ovat suunniteltu auttamaan käyttäjiä luomaan raportteja, kun taas raportointimoottorit ovat suunniteltu automatisoimaan itse raporttien luominen jonkin annetun lähdedatan perusteella.

Raportointimoottori on tyypillisesti taustalla toimiva, esimerkiksi palvelinsovellus, joka keskittyy raporttien prosessointiin ja renderöimiseen. Se huolehtii esimerkiksi tietojen hakemisesta, yhdistämisestä ja muotoilusta ennalta määritettyjen raporttipohjien ja ohjeiden perusteella. Se ei sisällä itse käyttöliittymää ja sen toimintoja käytetään ohjelmallisesti. Raportointityökalu sisältää taas usein jonkinlaisen käyttöliittymän, jonka avulla käyttäjä voi luoda uusia tai muokata olemassa olevia raportteja. Raportointityökalut voivat olla sisäänrakennettuja johonkin järjestelmään, esimerkiksi PLM-järjestelmään, mutta ne voivat olla myös ulkoisia ohjelmistoja, jotka pääsevät käsiksi analysoitavaan dataan esimerkiksi jonkin rajapinnan kautta. Koska raportointimoottori voi olla konfiguroitavissa loppukäyttäjien toimesta raportointityökalun avulla, tulee molempia komponentteja kehittäessä huomioida loppukäyttäjien yksilölliset ja monipuoliset tarpeet. \cite{adhi_performance_2019}

Bambang Prasetya Adhi ym. \cite{adhi_performance_2019} vertailevat tutkimusartikkelissa\textit{"Performance comparison of reporting engine birt, jasper report, and crystal report on the process business intelligence"}  suosituimpia kaupallisia (SAP Crystal Reports) ja avoimen lähdekoodin (BIRT ja Jasper Report) raportointityökaluja ja niiden alla toimivia raportointimoottoreita. Ahdi ym. käyttävät kokeellisia menetelmiä mitatakseen kolmea raportointityökalun osa-aluetta \cite{adhi_performance_2019}:
\begin{itemize}
\item Soveltuvuutta, esimerkiksi kuinka hyvin raportointimoottori tukee erilaisen lähdedatan käyttöä
\item Käytettävyyttä, joka ilmenee oppimisen helppoutena sekä toiminnallisuuden loogisuutena ja tehokkaana käyttönä
\item Tehokkuutta, joka mittaa itse järjestelmän tehokkuutta, esimerkiksi suoritusaikaa \cite{adhi_performance_2019}
\end{itemize}
Näitä osa-alueita arvioimalla voidaan tehdä perusteltuja päätöksiä raportointityökalun ja -moottorin valinnasta, joten nämä ovat tärkeitä seikkoja ottaa huomioon näiden suunnittelussa ja kehityksessä. 

\subsection{Raportointimoottorin rakenne ja prosessi} \label{Raportointimoottorin rakenne ja prosessi}

Prosessina raportointimoottori toimii kolmella tasolla: \textbf{data}-, \textbf{logiikka}- ja \textbf{esitystasolla}. \cite{he_design_2010}

\textbf{Datatasolla} raportointimoottori voi hakea dataa suoraan tietotokannasta tai esimerkiksi API\nomenclature[API]{API}{ohjelmointirajapinta, engl. Application Programming Interface}:n eli ohjelmointirajapinnan (engl. \textit{Application Programming Interface}) kautta. Raportointimoottorin tapauksessa ohjelmointirajapinta voi olla esimerkiksi hakurajapinta, jonka taustalla toimivan hakumoottorin avulla raportointimoottori voi hakea tarvitsemaansa dataa täsmällisemmin. Varsinkin PLM-järjestelmän kontekstissa hakumoottori on hyvin keskeinen osa PLM-järjestelmää. Datataso määrittelee siis mistä ja miten raportointimoottori hankkii lähtödataa sekä millaiset lähtötiedot sillä on koostaa raportti. \cite{he_design_2010} Näihin lähtötietoihin lukeutuu esimerkiksi mahdolliset raporttipohjat ja muut käyttäjän määrittämät asetukset.

\textbf{Logiikkatasolla} raportointimoottori jäsentelee lähtödataa ja suorittaa laskentaa toisella tasolla. Lähtödatan formaatti on usein historiallisesti ollut XML\nomenclature[XML]{XML}{merkintäkielien standardi ja tiedostomuoto, engl. Extensible Markup Language} (engl. \textit{Extensible Markup Language}) sen ollessa yksi internetin yleisimmin käytetyistä dataformaateista, mutta JavaScriptin yleistyttyä myös JSON\nomenclature[JSON]{JSON}{engl. JavaScript Object Notation, yksinkertainen tiedostomuoto tiedon välitykseen ja tallennukseen} (engl. \textit{JavaScript Object Notation}) noussut suosituksi tiedostomuodoksi. JSONin etuna on sen suora integraatio JavaScriptin yhteyteen sekä sen nopeus verrattuna vanhempaan XML-standardiin. \cite{nurseitov_comparison_nodate} Logiikkatason toteuttaman laskennan avulla lähtödatasta voidaan luoda esimerkiksi graafeja ja sekä taulukoida dataa, mikä mahdollistaa lasketut sarakkeet ja summat. Tämä voidaan nähdä raportointimoottorin ytimenä, sillä se tuottaa merkityksellistä dataa usein vaikeaselkoisesta lähdedatasta. \cite{he_design_2010} Lisäksi logiikkatason tulee jäsentää data siten, että se on mahdollista kirjoittaa tiedostoon esitystasolla.

\textbf{Esitystasolla} data kirjoitetaan tiedostoon, jolloin se voidaan tallentaa käyttäjäystävällisessä tiedostoformaatissa. \cite{he_design_2010} Suosituimpia tiedostoformaatteja ovat HTML-, Excel- ja PDF-tiedostoformaatit. \cite{he_design_2010} Esitystasolla raportin merkitys ilmenee käyttäjälle: lähtödata on esitetty helposti omaksuttavassa ja ymmärrettävässä muodossa, sekä raportti on luettavissa ja jaettavissa helposti yksittäisenä tiedostona.

\subsection{Raportointi PLM-järjestelmässä} \label{Raportointi PLM-järjestelmässä}

Kuten luvussa \ref{PLM-strategian hyödyt ja merkitys} todettiin, yksi PLM-järjestelmän hyödyistä on yksittäisten työntekijöiden työmäärän vähentäminen ja prosessien suoraviivaistaminen. Tietoteknisten järjestelmien etuna on varsinkin automoitu laskenta, jota hyödyntämällä voidaan vähentää inhimillisiä virheitä. \cite{niu_organizational_2021} \cite{rakovic_digital_2022} Automoitua laskentaa hyödynnetään erityisesti erilaisten raporttien muodostamisessa.

Koska PLM:n tarkoituksena on mahdollistaa koko tuotantoketjun yhteistyö asiakkaiden, kehittäjien, toimittajien ja valmistajien välillä tuotteen eri elinkaaren vaiheissa, \cite{bouhaddou_plm_2012} on tärkeää, että tuotetieto elinkaaren eri vaiheissa on dokumentoitavissa, analysoitavissa ja helposti jaettavissa. Koska PLM-järjestelmien tietomallit ovat usein ohjelmistokohtaisia ja harvemmin standardinomaisia\cite{SritiMohamed-Foued2012PTaS} on tärkeää, että tuotedataa on mahdollista viedä myös itse järjestelmän ulkopuolelle tallennettavaksi ja jaettavaksi.

PLM:n kontekstissa raporteilla tarkoitetaan tuotedataa kokoavia ja analysoivia kokonaisuuksia. Raportit voivat olla esimerkiksi digitaalisia PDF- tai Excel-dokumentteja, jotka kokoavat tuotetietoja ja suorittavat laskentaa visualisoimalla dataa esimerkiksi kuvaajin. Toinen PLM-järjestelmille tyypillinen raporttimuoto on interaktiiviset "kojelautoja" (engl. \textit{dashboards}), jotka kokoavat useita kuvaajia ja laskettuja arvoja yksittäiseen käyttäjäystävälliseen näkymään, tarjoamalla esimerkiksi jonkin Web-käyttöliittymän. Tässä tutkielmassa keskitytään enemmän raporttidokumenttien tuottamiseen ohjelmallisesti, mutta usein näihin digitaalisiin dokumentteihin on myös mahdollista upottaa kojelautamaisia ominaisuuksia, kuten kuvaajia ja tilastoja.

Raporttidokumenttien tuottamista varten monet PLM-järjestelmät tarjoavat PLM-ohjelmiston osana raportointityökalun, jonka tarkoituksena on kerätä ja analysoida dataa kokoamalla sitä dokumenttitiedostoformaatteihin ennalta määriteltyjen sääntöjen ja mallien mukaisesti. Lisäksi raportointityökalu voi tarjota jonkinlaisen käyttöliittymän raporttien muokkaamiseen ja konfigurointiin. Alan standardina dokumenttiformaateista raporttien tapauksessa lienee PDF-, Excel- ja HTML-pohjaiset raportit, sillä useimmat raportointityökalut tarjoavat raportteja näissä tiedostomuodoissa ja ne ovat myös tuttuja suurimmalle osalle ohjelmiston käyttäjistä. 

\subsection{PLM-järjestelmä toimintaympäristönä} \label{PLM-järjestelmä toimintaympäristönä}

Rohleder ym. \cite{rohleder_requirements_2014} käsittelee tutkimuksessaan \textit{"Requirements Engineering in Business Analytics for Innovation and Product Lifecycle Management"} vaatimusten määrittelyä liiketoimintatiedon hyödyntämisessä PLM-järjestelmissä. Rohleder ym. korostaa, että PLM-järjestelmissä toimitaan usein massadatan (engl. \textit{Big data}) \nomenclature[Big data]{Big data}{suom. massadata, erittäin suuret ja järjestämättömät jatkuvasti lisääntyvät tietomassat} parissa, sillä heidän tutkimuksen mukaan yksittäisen auton tuoterakenne voi koostua noin 120 tuhannesta yksittäisestä osasta, joilla jokaisella on tyypillisesti omia CAD-malleja (tietokoneavusteisen suunnitteluohjelman luomia tiedostoja)\nomenclature[CAD]{CAD}{engl. Computer Aided Design, tietokoneen käyttö suunnittelun apuvälineenä}, piirustuksia ja metadataa. Lisäksi tuotteen useat versiot ja variantit kasvattavat lopullista tietomäärää eksponentiaalisesti. Lisäksi PLM-järjestelmiä käyttää tyypillisesti useita työntekijöitä yrityksen eri osastoissa, jolloin näiden työnkulkujen erilaisuus lisää PLM-järjelmien datan kompleksisuutta entisestään. Kompleksisuus johtaa usein siihen, että valmiit raportointimoottorit, varsinkin esimerkiksi taloudelliseen raportointiin erikoistuneet, eivät välttämättä sovi sellaisenaan käytettäväksi PLM:n kontekstissa. \cite{rohleder_requirements_2014}

Kuten kappaleessa \ref{Osaluettelo PLM-järjestelmässä} todettiin, osaluettelot ovat keskeinen osa PLM-järjestelmään tallennetun datan esittämistä. Koska tuoteobjektit koostuvat osaluetteloista, myös PLM-järjestelmässä tuotteista koostettavat raportit perustuvat osaluetteloista kerättyyn lähtödataan. PLM-järjestelmän raportointimoottorit ovat siten erikoistuneita jäsentelemään ja kokoamaan hierarkkista dataa. \cite{rohleder_requirements_2014} PLM-järjestelmän tarjoamille raporteille on olennaista tuotteeseen ja sen kehitykseen liittyvät seikat, kuten esimerkiksi tuotteen osien toimittajien jakauma tai tuotteen muokkaushistoria. Lisäksi osaluettelon perusteella voidaan laskea yksittäisten osien summia rakenteessa tai esimerkiksi luoda raportteja tietyistä tuotteen osista, jotka täyttävät annetut kriteerit.

Mahdollisen raportointimoottorin ohella PLM-järjestelmät sisältävät usein hakumoottorin, jonka avulla voidaan etsiä tehokkaasti ja tarkasti tietokannasta annettujen kriteerien mukaisesti. Tiedon haku on yksi PLM-järjestelmän keskeisimmistä ominaisuuksista. \cite{enriquez_approach_2019} Raporttien muodostamisessa ulkoisen hakumoottorin hyödyntäminen vähentää itse raportointimoottorin kuormaa, jolloin raportointimoottorin toiminallisuuden kehittämisessä voidaan keskittyä enemmän itse laskentaan ja lisäarvon tuottamiseen. Täten PLM-järjestelmän tapauksessa lähtödatan hakeminen voi tapahtua esimerkiksi jonkin hakurajapinnan välityksellä, jolloin raportteja voidaan muodostaa tuoterakenteiden lisäksi esimerkiksi jonkin tietyn hakulausekkeen perusteella.

PLM-järjestelmien käyttäjät ovat tyypillisesti suhteellisen suuren mittakaavan teollisuusyrityksiä. Useissa tapauksissa myös tuoterakenteet ovat valtavia \cite{rohleder_requirements_2014}, joten raportointimoottorin tulee olla tarpeeksi tehokas ja optimoitu, jotta myös suurista tietorakenteista on mahdollista koostaa raportteja siedettävässä suoritusajassa. Raportointimoottorin logiikkatason lisäksi PLM-järjestelmien käyttäjillä on myös tarpeita muokata raporttien ulkoasua raportointimoottorin esitystasolla. Esimerkiksi yrityksen logojen ja raportin visuaalisen ilmeen muokkaaminen on olennainen osa taas raportointityökalun toiminnallisuutta. Koska raportointityökalun tulee olla sisällytetty saumattomasti muihin PLM-järjestelmän ominaisuuksiin hyvän käyttäjäkokemuksen varmistamiseksi, useat PLM-järjestelmiä tarjoavat yritykset käyttävät PLM-järjestelmissään tätä järjestelmää varta vasten kehitettyjä raportointityökaluja. Tarvetta varta vasten kehitetylle raportointityökalulle lisää myös mainittu PLM-datan kompleksisuus, koska olemassa olevat raportointityökaluratkaisut eivät välttämättä sovellu sellaisenaan toimimaan kompleksisen PLM-datan kanssa.
