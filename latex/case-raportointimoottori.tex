\chapter{Case-tapaus: Raportointimoottorin kehittäminen osaksi Sovelia PLM-järjestelmää} \label{Case-tapaus: Raportointimoottorin kehittäminen osaksi Sovelia PLM-järjestelmää}

\section{Sovelia PLM} \label{PLM-järjestelmät}

\textit{Sovelia PLM on kaupallinen PLM-järjestelmä...} (selostus yleisellä tasolla kehitettävän raportointityökalun toimintaympäristöstä)

Kehitettävän raportointityökalun toimintaympäristönä toimii kaupallinen PLM-järjestelmä, Sovelia PLM. Sovelia PLM:n kehitys on alkanut yli 30 vuotta sitten ja sen kehitys jatkuu aktiivisesti edelleen. Kuten muutkin PLM-järjestelmät, Sovelia PLM pyrkii ratkaisemaan valmistusalan yritysten haasteita liittyen tuotteen datan hallintaan sen koko elinkaaren ajan.\cite{soveliaAboutSovelia} Sovelia PLM:n erityispiirteenä on sisältämät valmiiksi konfiguroidut \textit{"templatet"} eli valmiit mallit objekteille ja prosesseille, jotka ovat muokattavissa asiakkaan tarpeiden mukaan. Lisäksi malleihin kuuluu muita valmiiksi konfiguroituja työkaluja että alalla hyväksi todettuja prosesseja.\cite{soveliaSoveliaGetting}

Sovelia PLM koostuu objekteista ja objektilinkeistä. Objekteja voidaan määritellä niiden attribuuttien avulla. Objektit voivat olla osia, piirrustuksia, dokumenttilinkkejä tai linkkejä muihin toimintohin. Objektien linkkien avulla osaluettelot muodostuvat, jotka ovat taas tärkeitä raportoinnin kannalta.\cite{soveliaSoveliaGetting}


\section{Nykyiset raportointimoottorit}

\textit{Alustus ja esittely nykyisiin raportointityökaluihin, niiden ominaisuuksiin ja käytettyihin tekniikoihin.}
