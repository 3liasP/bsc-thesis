\chapter{Raportointi ja PLM järjestelmät} \label{Raportointi ja PLM järjestelmät}

\section{PLM-strategia ja PLM-järjestelmät} \label{PLM-järjestelmät}

Laajempana käsitteenä tuotteen elinkaaren hallinta eli PLM voidaan nähdä yrityksen strategiana hallita tuotetietoja. PLM strategiana koostuu tuotteista, organisaatioista, työmenetelmistä, prosesseita, ihmisistä ja lopulta myös tietoteknisestä elinkaaren hallintajärjestelmästä. \cite{alemanni_key_2008}

\subsubsection{PLM strategian hyödyt} \label{PLM strategian hyödyt}

PLM strategian hyötyjä on käsitelty laajasti \cite{alemanni_key_2008} \cite{rivest_product_2012}. Strategian hyödyt voidaan jakaa kahteen osa-alueeseen: lyhyen ja pitkän aikavälin hyötyihin. PLM-järjestelmien tarkoituksena on taas mahdollistaa tämän strategian käyttöönottaminen. Näiden järjestelmien pääasiallisena tarkoituksena on koota tietoa yrityksen tuotteiden koko elinkaaren vaiheista keskitettyyn tietojärjestelmään. Tämä mahdollistaa laajojen tuotekantojen johdonmukaisen ja keskitetyn hallinnan yhteistyössä yrityksen eri osastojen ja kumppaneiden välillä.

\subsubsection{Lyhyen aikavälin hyödyt} \label{Lyhyen aikavälin hyödyt}

Lyhyellä aikavälillä PLM-strategia ja PLM-järjestelmän käyttöönotto voi vähentää aikaa, jota käytetään jokapäiväisten tehtävien suorittamiseen. Strategian avulla yrityksen tuotetiedot ovat keskitetysti saatavilla, eikä ajantasaisia tietoa tarvitse kysellä eri osastojen välillä. Lisäksi tuotteiden rakenteiden ymmärtäminen ja visualisointi helpottuu PLM-järjestelmän käyttöönoton myötä. Rakenteen ymmärystä ja jakamista voidaan parantaa entisestään myös PLM-järjestemän raportoinnilla.

\subsubsection{Pitkän aikavälin hyödyt} \label{Pitkän aikavälin hyödyt}

Kesken... 

\section{Osaluettelo PLM-järjestelmän sydämenä} \label{Raportointi}

Yksi PLM-järjestelmän tärkeimmistä toiminnallisuuksista on tuotteen osaluettelon (BOM\nomenclature[BOM]{BOM}{engl. Bill of Materials, osaluettelo, tuoterakenne}) esittäminen organisoidusti. \cite{david_what_2016} Yksinkertaisuudessaan osaluettelo on lista kaikista osista, joita tarvitaan tuotteet valmistamiseen. Osaluettelossa jokainen yksittäiseen osaan voidaan liittää useita tietokenttiä kuten valmistaja, versio, materiaali, määärä. Osaluettelo koostuu usein hierarkisesti osakokoonpanoista, välikokoonpanoista, osakomponenteista ja yksittäisistä osista, eli se kerää dataa siitä, kuinka eri komponentit ovat riippuvaisia toisistaan. Osaluetteloa voidaan käyttää viestintään esimerkiksi valmistuskumppanien välillä tai se voidaan rajoittaa yhteen tuotantoyksikköön.  \cite{jones_visualizing_2023}


