\chapter{Yhteenveto} \label{Yhteenveto}

\subsubsection{Tutkimuskysymykset}
\begin{itemize}
\item Miten PLM-järjestelmät ja raportointityökalut toimivat yhdessä?
\item Mitä tulee ottaa huomioon raportointityökalua kehittäessä osaksi PLM-järjestelmää?
\item Millaisia ovat nykyiset raportointityökalut?
\end{itemize}


Tämän tutkielman ensimmäisenä tutkimuskysymyksenä oli selvittää, miten PLM-järjestelmät ja raportointityökalut toimivat yhdessä. Kysymystä taustoitettiin luvussa \ref{Raportointi ja PLM} tarkastelemalla PLM-järjestelmiä ja raportointia erikseen sekä niiden yhteyttä. Tutkielman motiiveja taustoitettiin perehtymällä PLM-strategian hyötyihin ja PLM:n yhteyteen liiketoimintatiedon hyödyntämisessä. Aiempaan kirjallisuuten vedoten todettiin, että PLM-järjestelmä olennainen osa PLM-strategian hyödyntämistä käytännössä.\cite{alemanni_key_2008} Osaluettelon (engl. \textit{Bill of Materials, BOM}) merkitystä korostettiin yhtenä PLM-järjestelmän tärkeimmistä toiminnallisuuksista\cite{david_what_2016} luvussa \ref{Osaluettelo PLM-järjestelmässä}. Samassa luvussa perusteltiin, että osaluettelot ovat tärkeä osa PLM-järjestelmän raportointia \cite{german_challenge_2016}. Aiemman tutkimuksen perusteella voidaan myös todeta, että PLM-strategiassa liiketoimintatiedon hyödyntäminen korostuu etenkin osaluetteloista koostuvan tuotetiedon hyödyntämisenä. \cite{bayro-corrochano_preliminary_2014}  PLM-strategian ja -järjestelmän taustoittamisen jälkeen perehdyttiin raportointimoottoreihin ja työkaluihin luvussa \ref{Raportointimoottorit ja -työkalut}. Aluksi määriteltiin raportointityökalun ja -moottorin käsitteet, jonka jälkeen tarkasteltiin raportointimoottorin rakennetta ja prosessia. Luvussa \ref{Raportointimoottorin rakenne ja prosessi} raportointimoottori jaettiin aiemman tutkimuksen tavan mukaisesti kolmelle tasolle: data, -logiikka ja esitystasolle. \cite{he_design_2010}.

Raportointimoottorin tasojen tarkastelun jälkeen pyrittiin vastaamaan ensimmäiseen tutkimuskysymykseen PLM-järjestelmien ja raportointityökalujen yhteydestä. Tarkastelemalla ensin raportointia PLM-järjestelmässä todettiin, että tässä tutkielmassa keskitytään digitaalisten raporttidokumenttien tuottamiseen ohjelmallisesti. PLM-järjestelmän todettiin olevan haasteellinen ja yksilöllinen toimintaympäristö aiemman tutkimuksen tarkastelun perusteella, sillä valmiit raportointimoottorit, eivät välttämättä sovi sellaisenaan käytettäväksi usein massadatan kaltaisen PLM-datan tapauksessa.\cite{rohleder_requirements_2014} Lisäksi huomattiin, että PLM-järjestelmän raportointimoottorit
