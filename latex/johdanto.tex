\chapter{Johdanto} \label{Johdanto}

Tietotekniikan avulla voidaan tehostaa ja helpottaa työnteon tuottavuutta, kun samaan tehtävään käytetty aika vähenee \cite{rakovic_digital_2022}. Tämän tutkielman kontekstissa raportoinnilla tarkoitetaan yrityksen tapaa ja toimintoja kerätä, prosessoida, tallentaa ja esittää tietoa yrityksen sisäisesti, että ulkoisesti. Raportoinnin ydinajatuksena on tuottaa tietoa muodossa, joka on helposti ymmärrettävissä ja jaettavissa \cite{glockner_reports_2022}. Raporttien tyyppi voi vaihdella, mutta tässä tutkielmassa keskitytään erityisesti taulukko- ja kuvaajapohjaisiin raporttidokumentteihin.
 
Tietotekniikan hyödyntäminen raportointiin hyvin luonnollista, mikäli raportit ovat digitaalisesti tuotettuja dokumentteja, joiden kokoaminen vaatii jonkinlaista laskentaa. Raportointidatan kerääminen ja jäsenteleminen manuaalisesti on hyvin vaivalloista ja hidasta, mikä vuoksi useat tietojärjestelmät tarjota raportointityökaluja, jotka automatisoivat kaikki nämä vaiheet. Raportointityökalujen avulla olemassa olevasta suuresta määrästä dataa voidaan tuottaa selkeä ja jäsennelty esitys, joka kokoaa lähdedatan tärkeimmät seikat helposti yhdellä silmäyksellä omaksuttavaan muotoon.\cite{adhi_performance_2019}

Tämän työn tarkoituksena oli toteuttaa raportointityökalu osaksi Sovelia PLM -järjestelmää. Sovelia PLM on kaupallinen tuotteen elinkaaren hallintajärjestelmä \nomenclature[PLM]{PLM}{engl. Product Lifecycle Management, tuotteen elinkaaren hallinta} (engl. \textit{Product Lifecycle Management}, PLM)\cite{soveliaAboutSovelia}. PLM-järjestelmän pääasiallisena tarkoituksena on koota tietoa yrityksen tuotteiden koko elinkaaren vaiheista keskitettyyn tietojärjestelmään \cite{alemanni_key_2008}. Tämä keskitetty tietojärjestelmä on käytettävissä yrityksen eri työryhmien ja liiketoimintajärjestelmien välillä, minkä tarkoituksena on vähentää virheellisten tuotetietojen aiheuttamia turhia kustannuksia sekä viivästyksiä ja siten nopeuttaa yrityksen prosessia saada kehitettävä tuote markkinoille. \cite{alemanni_key_2008}

Raportointityökalu voidaan nähdä yhtenä PLM järjestelmälle lisäarvoa tuottavana ominaisuutena \cite{german_challenge_2016}. Luotettavan, tehokkaan ja mukautuvan raportointityökalun avulla PLM-järjestelmä voi tuottaa enemmän lisäarvoa sen käyttäjille tarjoamalla mahdollisuuden jakaa, tallentaa ja analysoida tuotedataa eri tiedostoformaateissa sekä yrityksen sisäisten työryhmien että ulkoisten toimijoiden välillä.\cite{german_challenge_2016} PLM-järjestelmässä raportointi osaluetteloihin \nomenclature[BOM]{BOM}{Bill of Materials, osaluettelo, tuoterakenne}(engl. \textit{Bill of Materials, BOM}) tallennetun datan perusteella on avainasemassa \cite{german_challenge_2016}. PLM-järjestelmiä käyttävillä yrityksillä on tyypillisesti suuria määriä tuotetietoja ja syviä tuoterakenteista koostuvia osaluetteloita \cite{rohleder_requirements_2014}, jolloin myös raportoinnin suorituskykyvaatimukset korostuvat.

PLM-järjestelmien tietomallit voidaan jakaa dokumentti- ja relaatiodatapohjaisiin tietorakenteisiin. \cite{david_what_2016} Koska Sovelia PLM -järjestelmä perustuu relaatiodatapohjaiseen tietomalliin, myös tässä tutkielmassa käsitellään PLM-järjestelmän raportointia nimenomaan relaatiodatan pohjalta.

Kehitettävä uusi raportointityökalu integroituu osaksi Sovelian nykyistä lähdekoodia ja sen palvelinkomponentteja. Ohjelmakokonaisuus koostu palvelinkomponentista, raportointimoottorista, joka tuottaa raporttitiedoston raportoinnin kohteena olevasta objektista, sekä konfigurointityökalusta, jonka avulla pääkäyttäjä voi muokata raporttien ulkonäköä ja rakennetta.

\subsubsection{Tutkimuskysymykset}
\begin{itemize}
\item[\textbf{TK1}] Miten PLM-järjestelmät ja raportointityökalut toimivat yhdessä?
\item[\textbf{TK2}] Millaisia ovat nykyiset raportointityökalut?
\item[\textbf{TK3}] Mitä tulee ottaa huomioon raportointityökalua kehittäessä osaksi PLM-järjestelmää?
\end{itemize}

\subsubsection{Tutkimusmenetelmät}
\begin{itemize}
\item Kirjallisuusanalyysi
\item Tapaustutkimus uuden raportointityökalun kehittämisestä
\end{itemize}

Jotta kirjallisuusanalyysi tutkimuskysymysten perusteella voitaisiin toteuttaa, tulee ensin tarkastella näiden hakumenetelmiä, joilla tämän tutkimuksen lähteet ovat kerätty. Yleisellä tasolla käytetyt lähteet voidaan jakaa akateemisiin ja yritysten tuottamiin julkaisuihin. Akateemiset tutkimukset löydettiin pääasiallisesti hyödyntämällä Turun Yliopiston kirjaston \textit{UTU Volter}-hakumoottoria, joka kerää Turun Yliopiston kirjaston aineistot yhteen tietokantaan. Tutkimusta aiheesta löydettiin myös \textit{Google Scholarin} avustuksella. 

% HAKUSANAT %

Alla olevassa kaaviossa on eritelty lähteitä ja niiden lukumääriä tarkemmin.

% KUVIO %

Tutkielman luvussa \ref{Raportointi ja PLM} hyödynnetään pääasiallisesti akateemisia lähteitä, jotta voidaan luoda ymmärrys PLM-strategiasta, -järjestelmistä sekä näiden yhteydestä raportointiin. Perehdymme ensin luvussa \ref{PLM-strategia ja PLM-järjestelmät} PLM:n merkitykseen ja sen ominaispiirteisiin. Käytämme aiempaa kirjallisuutta perustelemaan PLM:n merkityksellisyyttä ja sen yhteyttä liiketoimintatiedon hyödyntämiseen, minkä tehokas raportointi mahdollistaa.

Luvussa \ref{Raportointimoottorit ja -työkalut} määritellään akateemisen kirjallisuuden avulla raportointimoottorin ja -työkalun käsitteet ja rakenne. Tämän lisäksi perehdymme raportoinnin ja PLM-järjestelmän yhteyteen. Selvitämme myös, millainen PLM-järjestelmä on toimintaympäristönä ja millaisia niiden käyttäjät ovat.

Tutkielman luvussa \ref{Tapaus: Sovelia PLM:n raportointityökalu} esitteleme tapaustutkimuksen toimintaympäristön, Sovelia PLM:n, ominaisuuksia ja erityispiirteitä pohjautuen ohjelmiston dokumentaatioon. Tarkastelemme Sovelia PLM:n vanhaa raportointityökalua ja reflektoimme sen asettamia haasteita uuden raportointityökalun kehittämiselle. Tämän jälkeen perehdymme pääasiallisesti PLM-ohjelmistoja tarjoavien yritysten julkaisuihin PLM:n ja raportoinnin yhteydestä, josta päädymme tarkastelemaan kuutta olemassa olevaa raportointityökalua ja niiden ominaisuuksia. Nämä kuusi raportointityökalua valittiin, sillä niiden ominaisuudet ja toimintaympäristöt vastasivat uutta kehitettävää raportointityökalua.

Lopulta vertaamme havaintojamme kuudesta valitusta kehitettävän uuden raportointityökalun ominaisuuksiin ja erityispiirteisiin. Tämän perusteella pyrimme toteamaan, mitkä lähestymistavat soveltuivat Sovelia PLM:n kontekstiin ja mitkä eivät.

